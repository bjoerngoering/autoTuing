\clearpage
\section{User-Directed Autotuning}
While fully automated autotuning can be feasible in some cases, many
applications need users' expertise to obtain significant performance improvements. 
We revisit the SMG2000 benchmark in this section to see how one can use our autotuning system to have a user-directed empirical tuning process.
%-------------------------------------
\subsection{Manual Code Triage}
The SMG2000 kernel (shown in Fig.~\ref{Fig:smg2000kernel}) that is automatically identified by our simple code triage may not be the best tuning target. 
It is actually only a portion of a bigger computation kernel that is very
hard to be automatically identified. 
Also, the bigger kernel has a very complex form which impedes most
compilers or tools for further analyses and transformations. 
With the help from Rich Vuduc, who is a former postdoc with us, we manually transform the code (via code specification) to obtain a representative kernel which captures the core computation algorithm of the benchmark. 
We then put the outlining pragma (\lstinline{#pragma rose_outline}) before
the new kernel and invoked the ROSE outliner to separate it out into a new
source file, as shown below.

\lstset{language=C, basicstyle=\scriptsize}
\begin{lstlisting}
#include "autotuning_lib.h"
static double time1, time2;
void OUT__1__6119__(void **__out_argv);
typedef int hypre_MPI_Comm;
typedef int hypre_MPI_Datatype;
typedef int hypre_Index[3];

typedef struct hypre_Box_struct
{
   hypre_Index imin;
   hypre_Index imax;
} hypre_Box;

typedef struct hypre_BoxArray_struct
{
   hypre_Box *boxes;
   int size;
   int alloc_size;
} hypre_BoxArray;

typedef struct hypre_RankLink_struct
{
   int rank;
   struct hypre_RankLink_struct *next;
} hypre_RankLink;

typedef hypre_RankLink *hypre_RankLinkArray[3][3][3];

typedef struct hypre_BoxNeighbors_struct
{
   hypre_BoxArray *boxes;
   int *procs;
   int *ids;
   int first_local;
   int num_local;
   int num_periodic;
   hypre_RankLinkArray *rank_links;
} hypre_BoxNeighbors;

typedef struct hypre_StructGrid_struct
{
   hypre_MPI_Comm comm;
   int dim;
   hypre_BoxArray *boxes;
   int *ids;
   hypre_BoxNeighbors *neighbors;
   int max_distance;
   hypre_Box *bounding_box;
   int local_size;
   int global_size;
   hypre_Index periodic;
   int ref_count;
} hypre_StructGrid;

typedef struct hypre_StructStencil_struct
{
   hypre_Index *shape;
   int size;
   int max_offset;
   int dim;
   int ref_count;
} hypre_StructStencil;

typedef struct hypre_CommTypeEntry_struct
{
   hypre_Index imin;
   hypre_Index imax;
   int offset;
   int dim;
   int length_array[4];
   int stride_array[4];
} hypre_CommTypeEntry;

typedef struct hypre_CommType_struct
{
   hypre_CommTypeEntry **comm_entries;
   int num_entries;
} hypre_CommType;

typedef struct hypre_CommPkg_struct
{
   int num_values;
   hypre_MPI_Comm comm;
   int num_sends;
   int num_recvs;
   int *send_procs;
   int *recv_procs;
   hypre_CommType **send_types;
   hypre_CommType **recv_types;
   hypre_MPI_Datatype *send_mpi_types;
   hypre_MPI_Datatype *recv_mpi_types;
   hypre_CommType *copy_from_type;
   hypre_CommType *copy_to_type;
} hypre_CommPkg;

typedef struct hypre_StructMatrix_struct
{
  hypre_MPI_Comm comm;
  hypre_StructGrid *grid;
  hypre_StructStencil *user_stencil;
  hypre_StructStencil *stencil;
  int num_values;
  hypre_BoxArray *data_space;
  double *data;
  int data_alloced;
  int data_size;
  int **data_indices;
  int symmetric;
  int *symm_elements;
  int num_ghost[6];
  int global_size;
  hypre_CommPkg *comm_pkg;
  int ref_count;
} hypre_StructMatrix;

void OUT__1__6119__(void **__out_argv)
{
  static int counter =0 ;
  hypre_StructMatrix *A =  *((hypre_StructMatrix **)(__out_argv[20]));
  int ri =  *((int *)(__out_argv[19]));
  double *rp =  *((double **)(__out_argv[18]));
  int stencil_size =  *((int *)(__out_argv[17]));
  int i =  *((int *)(__out_argv[16]));
  int (*dxp_s)[15UL] = (int (*)[15UL])(__out_argv[15]);
  int hypre__sy1 =  *((int *)(__out_argv[14]));
  int hypre__sz1 =  *((int *)(__out_argv[13]));
  int hypre__sy2 =  *((int *)(__out_argv[12]));
  int hypre__sz2 =  *((int *)(__out_argv[11]));
  int hypre__sy3 =  *((int *)(__out_argv[10]));
  int hypre__sz3 =  *((int *)(__out_argv[9]));
  int hypre__mx =  *((int *)(__out_argv[8]));
  int hypre__my =  *((int *)(__out_argv[7]));
  int hypre__mz =  *((int *)(__out_argv[6]));
  int si =  *((int *)(__out_argv[5]));
  int ii =  *((int *)(__out_argv[4]));
  int jj =  *((int *)(__out_argv[3]));
  int kk =  *((int *)(__out_argv[2]));
  const double *Ap_0 =  *((const double **)(__out_argv[1]));
  const double *xp_0 =  *((const double **)(__out_argv[0]));

  at_begin_timing(); // begin timing

  for (si = 0; si < stencil_size; si++)
    for (kk = 0; kk < hypre__mz; kk++)
      for (jj = 0; jj < hypre__my; jj++)
        for (ii = 0; ii < hypre__mx; ii++)
          rp[(ri + ii) + (jj * hypre__sy3) + (kk * hypre__sz3)] -= 
             Ap_0[ii + (jj * hypre__sy1) + (kk * hypre__sz1) +  A -> data_indices[i][si]] *
             xp_0[ii + (jj * hypre__sy2) + (kk * hypre__sz2) + ( *dxp_s)[si]];
  at_end_timing(); //end timing

   *((int *)(__out_argv[2])) = kk;
   *((int *)(__out_argv[3])) = jj;
   *((int *)(__out_argv[4])) = ii;
   *((int *)(__out_argv[5])) = si;
   *((double **)(__out_argv[18])) = rp;
   *((hypre_StructMatrix **)(__out_argv[20])) = A;
}

\end{lstlisting}

As we can see, the new kernel directly and indirectly depends on some user defined types. 
The ROSE outliner was able to recursively find and copy them into the new
file in a right order.
%The header (autotuning\_lib.h) provides prototypes of timing and checkpointing/restarting 
%-------------------------------------------------------------
\subsection{Parameterized ROSE Loop Translators}
ROSE provides several standalone executable programs
(loopUnrolling, loopInterchange, and loopTiling under ROSE\_INSTALL/bin) for loop
transformations.
So autotuning users can use them via command
lines with abstract handles to create desired kernel variants.
Detailed information about the parameterized loop translators can be found
in Chapter 50 of the \htmladdnormallink{ROSE
Tutorial}{http://www.rosecompiler.org/ROSE_Tutorial/ROSE-Tutorial.pdf}.


These translators use ROSE's internal loop translation interfaces (declared within the SageInterface namespace). They are:

\begin{itemize}
\item \textit{bool loopUnrolling (SgForStatement *loop, size\_t
unrolling\_factor)}:
This function needs two parameters: one for the loop to be
unrolled and the other for the unrolling factor.
\item \textit{bool  loopInterchange (SgForStatement *loop, size\_t depth,
size\_t lexicoOrder)}:
The loop interchange function has three parameters, the first one to
specify a loop which starts a perfectly-nested loop and is to be
interchanged, the 2nd for the depth of the loop nest to be interchanged, and finally the
lexicographical order for the permutation.
\item \textit{bool  loopTiling (SgForStatement *loopNest, size\_t
targetLevel, size\_t tileSize)}:
The loop tiling interface needs to know the loop nest to be tiled, which
loop level to tile, and the tile size for the level.
\end{itemize}
For efficiency concerns, these functions only perform the specified
translations without doing any legitimacy check.
It is up to the users to
make sure the transformations won't generate wrong code.
%We will soon provide interfaces to do the eligibility check for each transformation.

Example command lines using the programs are given below:

{\scriptsize
\begin{verbatim}
# unroll a for statement 5 times. The loop is a statement at line 6 within an input file.

loopUnrolling -c inputloopUnrolling.C \
-rose:loopunroll:abstract_handle "Statement<position,6>" -rose:loopunroll:factor 5

# interchange a loop nest starting from the first loop within the input file,
# interchange depth is 4 and
# the lexicographical order is 1 (swap the innermost two levels)

loopInterchange -c inputloopInterchange.C -rose:loopInterchange:abstract_handle \
"ForStatement<numbering,1>" -rose:loopInterchange:depth 4 \
-rose:loopInterchange:order 1

# tile the loop with a depth of 3 within the first loop of the input file
# tile size is 5

loopTiling -c inputloopTiling.C -rose:loopTiling:abstract_handle \
"ForStatement<numbering,1>" -rose:loopTiling:depth 3 -rose:loopTiling:tilesize 5

\end{verbatim}
}

%-------------------------------------------------------------
\subsection{Connecting to the Search Engine}
We applied several standard loop optimizations to the new kernel.
They are, in the actual order applied,
loop tiling on i, j and k levels (each level has a same tiling size from 0
to 55 and a stride of 5),
loop interchange of i, j, k and si levels (with a lexicographical permutation order ranging from 0 to 4! -1),
and finally loop unrolling on the innermost loop only.
For all optimizations, a parameter
value of 0 means no such optimization is applied.
So the total search space has 14400 ($12*4!*50$) points.

The bash script used by the GCO search engine to conduct point evaluation is given blow. 
Note that a transformation command needs to consider previous transformations' side effects on the kernel. 
We also extended GCO to accept strides (using \textit{\#ST} in the script) for dimensions of search space.  
{\scriptsize
\begin{verbatim}  
$ cat eval_smg_combined
#!/bin/bash
#DIM    3
#LB     0 0 0
#UB     55 23 49
#ST     5 1 1

# number of executions to find the best result for this variant
ITER=3

# command line validation
# should have x parameters when calling this script
# x= number of dimensions for each point
if [ "$3" = "" ]; then
  echo "Fatal error: not enough  parameters are provided for all search dimensions"
  exit
fi

# convert points to transformation parameters
# Not necessary in this example

# target application path
APP_PATH=/home/liao6/svnrepos/benchmarks/smg2000
KERNEL_VARIANT_FILE=OUT__1__6119__.c
# remove previous variant of the kernel and result
/bin/rm -f $APP_PATH/struct_ls/$KERNEL_VARIANT_FILE /tmp/peri.result $APP_PATH/*.so *.so

# ------------ tiling i, j, k ----------------------
# first tiling is always needed.
loopTiling -c $APP_PATH/struct_ls/OUT__1__6119__perfectNest.c  -rose:loopTiling:abstract_handle "ForStatement<numbering,1>"\
 -rose:loopTiling:depth 4 -rose:loopTiling:tilesize $1 -rose:output $KERNEL_VARIANT_FILE

if [ $1 -ne 0 ]; then
loopTiling -c $KERNEL_VARIANT_FILE -rose:loopTiling:abstract_handle "ForStatement<numbering,1>" -rose:loopTiling:depth 4 \
-rose:loopTiling:tilesize $1 -rose:output $KERNEL_VARIANT_FILE loopTiling -c $KERNEL_VARIANT_FILE \
-rose:loopTiling:abstract_handle "ForStatement<numbering,1>" -rose:loopTiling:depth 4 -rose:loopTiling:tilesize $1 \
-rose:output $KERNEL_VARIANT_FILE

fi
# -------------- interchange si, k, j, i--------------
if [ $1 -ne 0 ]; then
loopInterchange -c $KERNEL_VARIANT_FILE -rose:output $KERNEL_VARIANT_FILE \
-rose:loopInterchange:abstract_handle 'ForStatement<numbering,4>' \
 -rose:loopInterchange:depth 4 -rose:loopInterchange:order $2
else
# No tiling happens, start from 1
loopInterchange -c $KERNEL_VARIANT_FILE -rose:output $KERNEL_VARIANT_FILE \
-rose:loopInterchange:abstract_handle 'ForStatement<numbering,4>' \
-rose:loopInterchange:depth 1 -rose:loopInterchange:order $2
fi

# ------------ unrolling innermost only -------------------
# generate a variant of the kernel using the transformation parameters
# unrolling the innermost level, must redirect to avoid mess up search engine
if [ $1 -ne 0 ]; then
loopUnrolling -c $KERNEL_VARIANT_FILE  -rose:loopunroll:abstract_handle "ForStatement<numbering,7>" \
-rose:loopunroll:factor $3 -rose:output $KERNEL_VARIANT_FILE > /dev/null 2>&1
else
loopUnrolling -c $KERNEL_VARIANT_FILE  -rose:loopunroll:abstract_handle "ForStatement<numbering,4>" \
-rose:loopunroll:factor $3 -rose:output $KERNEL_VARIANT_FILE > /dev/null 2>&1
fi

# build a .so for the kernel variant
# To redirect stdout to NULL is required
# since the search engine looks for stdout for the evaluation result
make -f makefile-lib  filename=OUT__1__6119__ > /dev/null 2>&1
cp OUT__1__6119__.so $APP_PATH/struct_ls/.

# generate a program to execute and timing the kernel
# Handled by ROSE

best_time="999999999.0"

# run the program multiple times
for (( i="1" ; i <= "$ITER" ; i = i + "1" ))
do

  # The tuned kernel will write timing information into /tmp/peri.result
   $APP_PATH/test/smg2000 -n 120 120 120 -d 3 > /dev/null 2>&1
  if [ $? -ne 0 ]; then
    echo "Error: program finishes abnormally!"
    exit 1
  else
    test -f /tmp/peri.result
    if [ $? -ne 0 ]; then
       echo "Error: The temp file storing timing information does not exist!"
       exit 1
    fi
    time=`tail -1 /tmp/peri.result | cut -f 1`
# select the smaller one
    best_time=`echo ${time} ${best_time} | awk '{print ($1 < $2) ? $1 : $2}'`
  fi
  # remove the temp file, the best time is kept by the script already
  /bin/rm -f  /tmp/peri.result
done

# report the evaluation result to the search engine
echo $best_time

\end{verbatim}
} % end scriptsize
%-------------------------------------------------------------
\subsection{Results}
%With the help from the checkpointing/restarting library and 
The new SMG2000 kernel is invoked thousands of times during a typical execution.
So instead of using checkpointing/restarting, we used a counter to reduce
the point evaluation time. The counter was set to be 1600, which means the
execution is terminated once the kernel is be called 1600 times.
By doing this, an exhaustive search using GCO
became feasible within 40 hours for an input data size of $120*120*120$.

The best performance was achieved at point (0,8,0), which means loop
interchange using the lexicographical number 8 (corresponding to an order of $[k,j,si,i]$) improved the
performance while tiling and unrolling did not help at all.
The best searched point achieved a 1.43x speedup for the kernel (1.18 for the whole
benchmark) when compared to the execution time using Intel C/C++ compiler v. 9.1.045 with option \textit{-O2}
on the Dell T5400 workstation.


